\documentclass[a4paper,12pt]{article}
\usepackage{amsmath}

\begin{document}

\begin{center}
{\large }コンピュータゼミ 2019 宿題
\end{center}

\section{1章}
私達の研究室では主にシステムやソフトウェアの
信頼性に関する研究を行っています.
主にそれらを確率論によってモデル化し,
解析することで信頼性の評価を行います.\\
\quad
具体的には以下のような確率過程を用いることが多いです.
\begin{itemize}
\item NHPP
\item CTMC
\end{itemize}
\section{2章}
卒業論文や原稿の作成のさいには\LaTeX
を使って文書を作成します.\LaTeX
は数式などを含むような
文章を綺麗に作成するための言語です.
\section{3章}
確率変数$X$が指数分布に従う時,
その分布関数$F_X(t)$と密度関数$f_x(t)$は,
\begin{align}
F_X(t)  &=  1-e^{-\lambda t} \label{F_X} \\
f_X(t)  &=  \lambda e^{-\lambda t} \label{f_x}
\end{align}
となる.またその期待値は定義により,
\begin{align}
E[X] &= \int_{0}^{\infty} t f_x(t) dt\nonumber\\
&= [(1-e^{-\lambda t}) t]^\infty_0-\int_{\infty}^{0} (1-e^{-\lambda t} dt)\nonumber\\
&= [(1-e^{-\lambda t}) t]^\infty_0-[t + \frac{1}{\lambda}e^{-\lambda t}]^\infty_0\nonumber\\
&= \frac{1}{\lambda}
\end{align}

となる.(extra 宿題:式(3)を導出してみよう
ヒント:部分積分)
\section{4章}
表をつくることもできます
\begin{center}
\begin{tabular}{|c|c|c|}
\hline
 1 & 2 & 3 \\\hline
 $\alpha$ & $\beta$ & $\gamma$ \\\hline
\end{tabular}
\end{center}
\end{document}
